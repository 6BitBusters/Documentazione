\section{Introduzione}
\subsection{Scopo del documento}
    Questo documento ha lo scopo di definire le norme e le linee guida operative
    per il team \textit{Six Bit Busters} nello sviluppo del progetto
    \textit{3Dataviz}.\\ In particolare, esso descrive i processi di lavoro,
    le modalità di collaborazione, gli standard di codifica e le pratiche
    di gestione della qualità che il team seguirà per garantire coerenza, 
    efficienza e qualità durante il ciclo di vita del prodotto. 
    Lo scopo è quello di fornire una struttura comune e procedure chiare, per 
    facilitare il lavoro in team e assicurare che tutti i membri operino in 
    linea con gli obiettivi e le specifiche concordate.

\subsection{Contesto}
    Il prodotto nasce dalla necessità dell'azienda SanMarco Informatica di 
    creare un'interfaccia web che prende il nome di 3Dataviz, per la presentazione di dati in grafici 3D interattivi
    e facilmente visitabili.
\subsection{Glossario}

\newpage

\section{Organizzazione del Team}
\subsection{Ruoli e responsabilità}
    I membri team \textit{Six Bit Busters} ricopriranno i ruoli principali 
    di un ciclo di vita del prodotto software, ovvero analista, 
    progettista, programmatore, verificatore, amministratore di sistema e responsabile. 
    \\Al fine di garantire una comprensione completa delle diverse fasi 
    e competenze richieste nello sviluppo di un progetto, i membri del team 
    ruoteranno periodicamente tra i ruoli ogni due settimane. Questa rotazione 
    periodica è finalizzata a scopi didattici, permettendo a ciascun membro di 
    acquisire una visione globale del ciclo di vita del prodotto e di sviluppare 
    abilità pratiche in ogni area.
\subsubsection{Analista}
    Colui che dialoga con il proponente e analizza l'esposizione che esso fornisce del 
    problema, nel nostro caso il progetto didattico, l'analista si occuperà di 
    analizzare a fondo il capitolato per estrarne i requisiti, per poi raffinarli 
    tramite incontri diretti con il proponente, durante lo svolgimento di questo 
    processo verrà redatto il documento \textit{Analisi dei requisiti}.
\subsubsection{Progettista}
    Colui che trasforma i requisiti in una soluzione, determinando una buona 
    architettura e definendo le scelte realizzative, prendono i requisiti 
    e li uniscono in una soluzione che sia la migliore possibile.
\subsubsection{Programmatore}
    Colui che si occupa di realizzare il design presentato dal progettista, 
    ovvero scrivere il codice quando esisterà una soluzione nota per rispondere 
    ai requisiti.
\subsubsection{Verificatore}
    Colui che mette in esame tutto ciò che viene fatto, dalla documentazione 
    al codice prodotto, controllando che non 
    ci siano errori o che qualcosa possa essere fatto meglio, con il supporto 
    di questa figura le attività potranno essere svolte secondo le attese.    
\subsubsection{Amministratore di sistema}
    Colui che agiste affinchè le attività siano agevolate da strumenti digitali, 
    che i prodotti siano persistenti e in luoghi disponibili, facilmente reperibili, definisce e 
    controlla l'ambiente di lavoro.
\subsubsection{Responsabile}
    Colui che condensa tutte le voci rappresentando una sola voce per rappresentare il progetto 
    al cliente, governa il team coordinandolo e gestendo le risorse, pianifica e coordina 
    le relazioni esterne, nel nostro caso con il proponente.\\
    \\
    Per l'analisi dei ruoli e la rendicontazione delle ore preventivate si faccia riferimento al documento \textit{Dichiarazione degli impegni}.

\subsection{Modalità di comunicazione}
    Il team ha scelto come canale di comunicazione interno asincrono un canale Telegram,
    mentre per le riunioni periodiche sincrone, una a settimana, ha optato per 
    l'utilizzo di un server Discord.\\
    Le comunicazioni con il fornitore verranno invece gestite in modo asincrono 
    utilizzando le mail e Google Chat, e in modo sincrono, preriodicamente ogni 
    XXX settimane, tramite riunioni in Google Meet.

\subsection{Gestione dei dubbi e conflitti}
    Nel caso sorgano dubbi, questi verranno risolti, in base all’urgenza, 
    tramite comunicazione nel canale Telegram o nella riunione interna 
    settimanale. \\Eventuali conflitti verranno discussi in sede di riunione 
    interna, con l’obiettivo di trovare una soluzione condivisa che consenta 
    al progetto di progredire secondo una visione comune.

\newpage

\section{Processi Primari}
I processi primari saranno presenti durante l'intero ciclo di vita del prodotto, 
considerato che il progetto assegnato è da considerarsi didattico, i processi 
primari non saranno considerati nella loro totalità ma solamente un sottoinsieme.\\
Infatti non ci dovremo occupare di installazione e manutenzione, ci fermeremo allo sviluppo.

\subsection{Fornitura}
\subsubsection{Descrizione}
La seguente sezione descrive le regole che il team si impegna a seguire 
per instaurare e mantenere una collaborazione proficua e costruttiva con il proponente, 
SanMarco Informatica.
\subsubsection{Scopo}
Il processo di fornitura, nei processi primari, si occupa di gestire le relazioni 
con il cliente o committente, assicurando che i requisiti del progetto siano 
compresi, rispettati e soddisfatti.
\subsubsection{Rapporto con il proponente}
Il gruppo \textit{Six Bit Busters} si impegna a mantenere il contatto con il proponente, 
organizzando incontri periodici sincroni sulla piattaforma Google Meet, o tramite comunicazioni 
asincrone con email o Google Chat.\\
Riteniamo che la comunicazione sia fondamentale per poter procedere secondo le aspettative 
richieste, e per essere sempre allineati sullo stato di avanzamento del progetto, in particolare 
ci aspettiamo che verranno toccati i seguenti temi:
\begin{itemize}
    \item Chiarimenti di dubbi o incomprensioni;
    \item Raccogliere riscontri sul lavoro svolto;
    \item Definire i requisiti che il prodotto dovrà soddisfare;
    \item Ricevere orientamento sulle tecnologie di riferimento per il progetto;
    \item Ottenere feedback nel caso il gruppo proponesse soluzioni alternative;  
\end{itemize}

