\newcounter{R}
\stepcounter{R}


\newcommand{\requirementsTable}[1]{

\renewcommand{\arraystretch}{1.5}
\rowcolors{2}{pari}{dispari}
\begin{longtable}{ %0.87
		>{\centering}M{0.15\textwidth} 
		>{\centering}M{0.5\textwidth}
		>{\centering}M{0.20\textwidth}
		>{\centering\arraybackslash}M{0.15\textwidth} 
		 }
	\rowcolorhead
	\headertitle{Requisito} &
	\centering \headertitle{Descrizione} &	
	\headertitle{Fonte}
	\endfirsthead	
	\endhead
	
	#1

\end{longtable}
\vspace{2em}

}

\newcommand{\SrcToReqTable}[1]{

\renewcommand{\arraystretch}{1.5}
\rowcolors{2}{pari}{dispari}
\begin{longtable}{ %0.87
		>{\centering}M{0.5\textwidth} 
		>{\centering\arraybackslash}M{0.20\textwidth} 
		 }
	\rowcolorhead
	\headertitle{Fonte} &
	\centering\headertitle{Requisito}
	\endfirsthead	
	\endhead
	
	#1

\end{longtable}
\vspace{2em}

}

\newcommand{\ReqToUCTable}[1]{

\renewcommand{\arraystretch}{1.5}
\rowcolors{2}{pari}{dispari}
\begin{longtable}{ %0.87
		>{\centering}M{0.5\textwidth} 
		>{\centering\arraybackslash}M{0.20\textwidth} 
		 }
	\rowcolorhead
	\headertitle{Requisito} &
	\centering\headertitle{Fonte}
	\endfirsthead	
	\endhead
	
	#1

\end{longtable}
\vspace{2em}

}

\newcommand{\requirementsSummaryTable}[1]{

\renewcommand{\arraystretch}{1.5}
\rowcolors{2}{pari}{dispari}
\begin{longtable}{ %0.87
		>{\centering}M{0.3\textwidth} 
		>{\centering}M{0.15\textwidth}
		>{\centering}M{0.15\textwidth}
		>{\centering}M{0.15\textwidth}
		>{\centering\arraybackslash}M{0.15\textwidth} 
		 }
	\rowcolorhead
	\headertitle{Tipologia} &
	\centering \headertitle{Obbligatorio} &	
	\headertitle{Desiderabile} &
	\headertitle{Opzionale} &
	\headertitle{Totale}
	\endfirsthead	
	\endhead
	
	#1

\end{longtable}
\vspace{2em}

}

\section{Requisiti}
\subsection{Introduzione}
Sono stati definiti dei requisiti codificati in base all’ambito di competenza e ad un numero seriale per
tenerne meglio traccia, inoltre nelle tabelle sottostanti sono fornite di descrizione e classificazione di ciascun
requisito. 
Per la codifica dei requisiti si faccia riferimento alle \textit{Norme di Progetto}
\subsection{Requisiti funzionali}
\requirementsTable{
    RF1.\arabic{R}\stepcounter{R} & L’utente deve poter visualizzare i dataset proposti   & UC1 \tabularnewline
    RF1.\arabic{R}\stepcounter{R} & L’utente deve poter visualizzare i dettagli dataset selezionato   & UC1.1\par UC1.1.1\par UC1.1.2\par UC1.1.3\par \tabularnewline
    RF1.\arabic{R}\stepcounter{R} & L’utente deve poter caricare il dataset selezionato   & UC1.2 \tabularnewline

    RF1.\arabic{R}\stepcounter{R} & L’utente deve poter visualizzare i dati del dataset selezionato in forma tabellare   & UC2 \tabularnewline
    RF3.\arabic{R}\stepcounter{R} & Le barre de grafico 3D devono avere una colorazione appropriata che rendano i valori più veloci da comprendere   & UC2 \tabularnewline
    
    RF1.\arabic{R}\stepcounter{R} & L’utente deve poter visualizzare i dati del dataset selezionato in forma di grafico tridimensionale a barre verticali   & UC3 \tabularnewline
    
    RF1.\arabic{R}\stepcounter{R} & L’utente deve poter visualizzare gli assi del grafico tridimensionale    & UC4 \tabularnewline
    RF1.\arabic{R}\stepcounter{R} & L’utente deve poter visualizzare l`asse globale X del grafico tridimensionale    & UC4.1 \tabularnewline
    RF1.\arabic{R}\stepcounter{R} & L’utente deve poter visualizzare l`asse globale Y del grafico tridimensionale    & UC4.2 \tabularnewline
    RF1.\arabic{R}\stepcounter{R} & L’utente deve poter visualizzare l`asse globale Z del grafico tridimensionale    & UC4.3 \tabularnewline
    
    RF1.\arabic{R}\stepcounter{R} & L’utente deve poter visualizzare la legenda del grafico tridimensionale    & UC5 \tabularnewline
 
    RF1.\arabic{R}\stepcounter{R} & L’utente deve poter effettuare un`azione di pan nell` ambiente, spostando la telecamera sugli assi locali Y e X   & UC6 \tabularnewline
    
    RF1.\arabic{R}\stepcounter{R} & L’utente deve poter ruotare la telecamera a suo piacimento per visualizzare meglio il grafico tridimensionale   & UC7 \tabularnewline
    RF1.\arabic{R}\stepcounter{R} & L’utente deve poter ruotare la telecamera attorno all`asse globale X   & UC7.1 \tabularnewline
    RF1.\arabic{R}\stepcounter{R} & L’utente deve poter ruotare la telecamera attorno all`asse globale Y   & UC7.2 \tabularnewline
    
    RF1.\arabic{R}\stepcounter{R} & L’utente deve poter muovere la telecamera all`interno dell` ambiente   & UC8 \tabularnewline
    RF1.\arabic{R}\stepcounter{R} & L’utente deve poter muovere la telecamera sull`asse globale X   & UC8.1 \tabularnewline
    RF1.\arabic{R}\stepcounter{R} & L’utente deve poter muovere la telecamera sull`asse globale Y   & UC8.2 \tabularnewline
    RF1.\arabic{R}\stepcounter{R} & L’utente deve poter muovere la telecamera sull`asse globale Z   & UC8.3 \tabularnewline
    
    RF1.\arabic{R}\stepcounter{R} & L’utente deve poter riposizionare la telecamera alla sua posizione iniziale   & UC9 \tabularnewline
    
    RF1.\arabic{R}\stepcounter{R} & L’utente deve poter visualizzare i valori del dataset selezionando le barre opportune verticali del grafico tridimensionale   & UC10 \tabularnewline
    
    RF1.\arabic{R}\stepcounter{R} & L’utente deve poter selezionare elementi rappresentanti i dati del dataset   & UC11 \tabularnewline
    RF1.\arabic{R}\stepcounter{R} & L’utente deve poter selezionare le barre verticali del grafico tridimensionale   & UC11.1 \tabularnewline
    RF1.\arabic{R}\stepcounter{R} & L’utente selezionando una barra verticale del grafico tridimensionale fa si che la corrispondente cella della tabella venga evidenziata   & UC11.1.1 \tabularnewline
    RF1.\arabic{R}\stepcounter{R} & L’utente deve poter selezionare le celle della tabella   & UC11.2 \tabularnewline
    RF1.\arabic{R}\stepcounter{R} & L’utente deve poter selezionare una cella della tabella, così facendo la corrispondente barra del grafico tridimensionale venga evidenziata    & UC11.2.1 \tabularnewline
    RF1.\arabic{R}\stepcounter{R} & L’utente deve poter selezionare una cella della tabella, così facendo la telecamera cambi posizione mettendo in primo piano la corrispondente barra del grafico tridimensionale   & UC11.2.2 \tabularnewline

    RF1.\arabic{R}\stepcounter{R} & Il sistema modifica la trasparenza delle barre del grafico secondo azioni dell'utente   & UC12 \tabularnewline
    RF1.\arabic{R}\stepcounter{R} & L’utente deve poter selezionare una barra la mette in evidenza andando a rendere piu` trasparenti quelle che hanno un valore maggiore o minore  & UC12.1 \tabularnewline
    RF1.\arabic{R}\stepcounter{R} & L’utente deve poter scegliere un numero "X" che verrà utilizzato per rendere piu` trasparenti le barre, i quali valori non rientrano tra i primi o ultimi X   & UC12.2 \tabularnewline
    RF1.\arabic{R}\stepcounter{R} & L’utente deve poter scegliere se rendere piu` trasparenti le barre con valori superiori o inferiori rispetto al valor medio globale   & UC12.3 \tabularnewline
    
    RF2.\arabic{R}\stepcounter{R} & L’utente deve poter di scegliere visualizzare il piano parallelo alla base che rappresenta il valor medio di un singolo elemento dell'asse   & UC13 \tabularnewline
    
    RF1.\arabic{R}\stepcounter{R} & L’utente riceve un errore se il caricamento di un dataset fallisce   & UC14 \tabularnewline
    }
\setcounter{R}{1}
\subsection{Requisiti qualitativi}
    
\requirementsTable{
    RQ1.\arabic{R}\stepcounter{R} & Il software deve essere sviluppato seguendo le metriche e il modello di qualità descritti nel documento \textit{Norme di Progetto}   & Decisione interna \tabularnewline
    RQ1.\arabic{R}\stepcounter{R} & Il software deve essere sviluppato pubblicando il codice sorgente sul repository Github dedicato   & Decisione interna \tabularnewline
    RQ1.\arabic{R}\stepcounter{R} & Il software deve essere sviluppato in modo tale da supportare grandi volumi di dati   & Capitolato \tabularnewline
    RQ1.\arabic{R}\stepcounter{R} & L'architettura del software deve permettere con agilità di poter aggiungere nuove funzionalità e possibilità di iterazione con il grafico    & Capitolato \tabularnewline
    RQ1.\arabic{R}\stepcounter{R} & La suite di test deve essere robusta per garantire ciò specificato in RQ4   & Capitolato \tabularnewline
    RQ2.\arabic{R}\stepcounter{R} & Il software deve essere testato attraverso test di tipo e2e   & Capitolato \tabularnewline  
}
\setcounter{R}{1}  
\subsection{Requisiti di vincolo}
\requirementsTable{
    RV1.\arabic{R}\stepcounter{R} & Il software deve essere sviluppato utilizzando Typescript come linguaggio di programmazione primario   & Decisione interna \tabularnewline
    RV1.\arabic{R}\stepcounter{R} & Il software deve essere sviluppato utilizzando la libreria React per la creazione di un interfaccia visiva sempice   & Decisione interna \tabularnewline
    RV1.\arabic{R}\stepcounter{R} & Il software deve essere sviluppato utilizzando la libreria Three.js   & Decisione interna \tabularnewline
    RV1.\arabic{R}\stepcounter{R} & Il software deve essere compatibile dalla versione 131 del browser Chrome  & Decisione interna \tabularnewline
    RV1.\arabic{R}\stepcounter{R} & Il software deve essere compatibile dalla versione 133 del browser Firefox  & Decisione interna \tabularnewline
    RV1.\arabic{R}\stepcounter{R} & Il software deve essere compatibile dalla versione 115 del browser Opera  & Decisione interna \tabularnewline
    RV1.\arabic{R}\stepcounter{R} & Il software deve essere compatibile dalla versione 131 del browser Microsoft Edge  & Decisione interna \tabularnewline
}
\newpage
\setcounter{R}{1}
\subsection{Requisiti prestazionali}
\requirementsTable{
    RP1.\arabic{R}\stepcounter{R} & Il software quando carica un dataset di dimensioni 100x100 non deve avere un tempo di caricamento superiore a 2 secondi & Decisione interna \tabularnewline
    RP1.\arabic{R}\stepcounter{R} & Il software quando carica un dataset di dimensioni 500x500 non deve avere un tempo di caricamento superiore a 5 secondi & Decisione interna \tabularnewline
    RP1.\arabic{R}\stepcounter{R} & Il software quando carica un dataset di dimensioni 1000x1000 non deve avere un tempo di caricamento superiore a 8 secondi & Decisione interna \tabularnewline
    RP2.\arabic{R}\stepcounter{R} & Il software quando carica un dataset di dimensioni maggiori di 1000x1000 non deve avere un tempo di caricamento superiore a 15 secondi & Decisione interna \tabularnewline
    RP1.\arabic{R}\stepcounter{R} & Il software non deve avere un tempo di risposta superiore a 5 secondi & Decisione interna \tabularnewline
}


\subsection{Tracciamento requisiti}

\subsubsection{Fonte - Requisiti}
\SrcToReqTable{
    Capitolato & RQ1.3\par RQ1.4\par RQ1.5\par RQ2.6\tabularnewline
    Decisione interna & RQ1.1\par
                        RQ1.2\par 
                        RV1.1\par 
                        RV1.2\par 
                        RV1.3\par 
                        RV1.4\par
                        RV1.5\par
                        RV1.6\par
                        RV1.7\par
                        RP1.1\par
                        RP1.2\par
                        RP1.3\par
                        RP2.4\par
                        RP1.5\par
                        \tabularnewline
    UC1         & RF1.1 \tabularnewline
    UC1.1       & RF1.2 \tabularnewline
    UC1.1.1     & RF1.2 \tabularnewline
    UC1.1.2     & RF1.2 \tabularnewline
    UC1.1.3     & RF1.2 \tabularnewline
    UC1.2       & RF1.3 \tabularnewline
    UC2         & RF1.4\tabularnewline
    UC2         & RF3.5\tabularnewline
    UC3         & RF1.6 \tabularnewline
    UC4         & RF1.7 \tabularnewline
    UC4.1       & RF1.8    \tabularnewline
    UC4.2       & RF1.9 \tabularnewline
    UC4.3       & RF1.10 \tabularnewline
    UC5         & RF1.11 \tabularnewline
    UC6         & RF1.12 \tabularnewline
    UC7         & RF1.13 \tabularnewline
    UC7.1       & RF1.14 \tabularnewline
    UC7.2       & RF1.15 \tabularnewline
    UC8         & RF1.16 \tabularnewline
    UC8.1       & RF1.17 \tabularnewline
    UC8.2       & RF1.18 \tabularnewline
    UC8.3       & RF1.19 \tabularnewline
    UC9         & RF1.20 \tabularnewline
    UC10        & RF1.21 \tabularnewline
    UC11        & RF1.22 \tabularnewline
    UC11.1      & RF1.23    \tabularnewline
    UC11.1.1    & RF1.24  \tabularnewline
    UC11.2      & RF1.25    \tabularnewline
    UC11.2.1    & RF1.26  \tabularnewline
    UC11.2.2    & RF1.27  \tabularnewline
    UC12        & RF1.28 \tabularnewline
    UC12.1      & RF1.29    \tabularnewline
    UC12.2      & RF1.30    \tabularnewline
    UC12.3      & RF1.31   \tabularnewline
    UC13        & RF2.32 \tabularnewline
    UC14        & RF1.33 \tabularnewline
    }

    \newpage
\subsubsection{Requisiti - Casi d'uso}
\ReqToUCTable{

    RF1.1& UC1 \tabularnewline 
    RF1.2 & UC1.1\par UC1.1.1\par UC1.1.2\par UC1.1.3\tabularnewline
    RF1.3 & UC1.2 \tabularnewline
    RF1.4 & UC2 \tabularnewline
    RF3.5 & UC2 \tabularnewline
    RF1.6 & UC3 \tabularnewline
    RF1.7 & UC4 \tabularnewline
    RF1.8 & UC4.1 \tabularnewline
    RF1.9 & UC4.2 \tabularnewline
    RF1.10 & UC4.3 \tabularnewline
    RF1.11& UC5 \tabularnewline
    RF1.12& UC6 \tabularnewline
    RF1.13& UC7 \tabularnewline
    RF1.14 & UC7.1 \tabularnewline
    RF1.15 & UC7.2 \tabularnewline
    RF1.16& UC8 \tabularnewline
    RF1.17 & UC8.1 \tabularnewline
    RF1.18 & UC8.2 \tabularnewline
    RF1.19 & UC8.3 \tabularnewline
    RF1.20& UC9 \tabularnewline
    RF1.21 & UC10 \tabularnewline
    RF1.22 & UC11 \tabularnewline
    RF1.23 & UC11.1 \tabularnewline
    RF1.24 & UC11.1.1 \tabularnewline
    RF1.25 & UC11.2 \tabularnewline
    RF1.26 & UC11.2.1 \tabularnewline
    RF1.27 & UC11.2.2 \tabularnewline
    RF1.28 & UC12 \tabularnewline
    RF1.29 & UC12.1 \tabularnewline
    RF1.30 & UC12.2 \tabularnewline
    RF1.31 & UC12.3 \tabularnewline
    RF2.32 & UC13 \tabularnewline
    RF1.33 & UC14 \tabularnewline

    RQ1.1 & Decisione interna\tabularnewline
    RQ1.2& Decisione interna\tabularnewline
    RQ1.3 & Capitolato\tabularnewline
    RQ1.4 & Capitolato\tabularnewline
    RQ1.5 & Capitolato\tabularnewline
    RQ2.6 & Capitolato\tabularnewline

    RV1.1 & Decisione interna\tabularnewline
    RV1.2 & Decisione interna\tabularnewline
    RV1.3 & Decisione interna\tabularnewline
    RV1.4 & Decisione interna\tabularnewline
    RV1.5 & Decisione interna\tabularnewline
    RV1.6 & Decisione interna\tabularnewline
    RV1.7 & Decisione interna\tabularnewline

    RP1.1 & Decisione interna\tabularnewline
    RP1.2 & Decisione interna\tabularnewline
    RP1.3 & Decisione interna\tabularnewline
    RP2.4 & Decisione interna\tabularnewline
    RP1.5 & Decisione interna\tabularnewline

}

\subsubsection{Riepilogo requisiti}
\requirementsSummaryTable{
    Funzionale & 32 & 1 & 1 & 34 \tabularnewline
    Di qualità & 5 & 1 & 0 & 6 \tabularnewline
    Di vincolo & 7 & 0 & 0 & 7 \tabularnewline
    Prestazionali & 4 & 1 & 0 & 5 \tabularnewline
    \bottomrule
    \textbf{Totale requisiti} & 48 & 3 & 1 & 52
    
}

