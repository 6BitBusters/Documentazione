\newcounter{M}

\newcommand{\MPCQ}[1]{
    \stepcounter{M}
    \subsubsection{MPCQ\arabic{M} - #1}
    }
    \newcommand{\MPDQ}[1]{
    \stepcounter{M}
    \subsubsection{MPDQ\arabic{M} - #1}
}

\section{Metriche di qualità}
\subsection{Introduzione}
Le metriche di qualità sono strumenti fondamentali per
valutare e migliorare l’efficacia e l’efficienza nello sviluppo del software. Queste metriche
forniscono indicatori oggettivi e misurabili che consentono di valutare la conformità agli
standard, identificare aree di miglioramento e monitorare la salute complessiva del processo
di sviluppo.
\subsection*{Codifica}
\begin{center}
    \textbf{M[Tipologia][Id numerico]}
\end{center}
dove:
\begin{itemize}
    \item \textbf{Tipologia}: indica il tipo della metrica:
    \begin{itemize}
        \item \textbf{PCQ}: per il processo;
        \item \textbf{PDQ}: per il prodotto.
    \end{itemize}
    \item \textbf{Id numerico}: indica un numero univoco incrementale separato per le due tipologie.
\end{itemize}

\subsection{Metriche per la qualità di processo}
Di seguito sono descritte le metriche di qualità di processo che il gruppo intende adottare.
\MPCQ{Planned Value (PV)}
\MPCQ{Actual Cost (AC)}
\MPCQ{Estimated at Completion (EAC)}
\MPCQ{Earned Value (EV)}
\MPCQ{Estimated to Complete (ETC)}
\MPCQ{Cost Variance (CV)}
\MPCQ{Schedule Variance (SV)}
\MPCQ{Budget Variance (BV)}
\MPCQ{Requirements Stability Index (RSI)}
\MPCQ{Gulpease Index (GI)}
\MPCQ{Metrics Met (MM)}
\MPCQ{Code Coverage (COC)}
\MPCQ{Unexpected Risks (UR)}


\setcounter{M}{0}
\subsection{Metriche per la qualità di prodotto}
Di seguito sono descritte le metriche di qualità di prodotto che il gruppo intende adottare.
\MPDQ{Met Requirements Percent (MRP)}
\MPDQ{Fault Density(FD)}
\MPDQ{Average Response Time (ART)}
\MPDQ{Loading Time (LT)}
\MPDQ{Ease Of Learning (EOL)}
\MPDQ{Cyclomatic Complexity (CYC)}
\MPDQ{Comment Density (CD)}
\MPDQ{Supported Browser(SB)}

