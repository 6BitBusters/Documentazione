\section{Standard per la qualità}

\subsection{ISO/IEC 9126}
Per definire, misurare e valutare il software prodotto, il gruppo adotta lo 
standard ISO/IEC 9126, che definisce un modello di qualità del software. Tale 
modello è organizzato in sei caratteristiche principali, 
ognuna delle quali è suddivisa in sotto-caratteristiche.

\subsubsection{Funzionalità}
La funzionalità misura la capacità del software di soddisfare i requisiti specificati e di fornire funzioni corrette.
\begin{itemize}
    \item \textbf{Appropriatezza}: capacità di eseguire i compiti richiesti;
    \item \textbf{Accuratezza}: capacità di fornire risultati corretti;
    \item \textbf{Interoperabilità}: capacità di interagire con uno o più sistemi specificati;
    \item \textbf{Sicurezza}: capacità di aderire a standard, convenzioni e regolamenti;
    \item \textbf{Conformità}: capacità di aderire a standard, convenzioni e regolamenti.
\end{itemize}

\subsubsection{Affidabilità}
L'affidabilità misura la capacità del software di svolgere correttamente il suo compito
mantenendo un un certo livello di prestazioni quando usato in determinate condizioni.
\begin{itemize}
    \item \textbf{Maturità}: capacità di evitare errori, malfunzionamenti e guasti;
    \item \textbf{Tolleranza agli errori}: capacità di mantenere un certo livello di prestazioni anche in presenza di errori;
    \item \textbf{Recuperabilità}: capacità di ripristinare un certo livello di prestazioni e di dati in seguito a un errore;
    \item \textbf{Conformità}: capacità di aderire a standard, convenzioni e regolamenti.
\end{itemize}

\subsubsection{Efficienza}
L'efficienza misura la capacità del software di fornire prestazioni appropriate rispetto alla quantità di risorse utilizzate.
\begin{itemize}
    \item \textbf{Comportamento rispetto al tempo}: capacità di fornire tempi di risposta adeguati;
    \item \textbf{Utilizzo delle risorse}: capacità di utilizzare un numero di risorse adeguato;
    \item \textbf{Capacità}: capacità di aderire a standard, convenzioni e regolamenti.
\end{itemize}

\subsubsection{Usabilità}
L'usabilità misura la capacità del software di essere compreso, appreso, usato e attraente per l'utente.
\begin{itemize}
    \item \textbf{Comprensibilità}: capacità di essere compreso;
    \item \textbf{Apprendibilità}: capacità di essere appreso;
    \item \textbf{Operabilità}: capacità di essere usato;
    \item \textbf{Attrattività}: capacità di essere attraente.
\end{itemize}

\subsubsection{Manutenibilità}
La manutenibilità misura la capacità del software di essere modificato, includendo correzioni, miglioramenti o adattamenti.
\begin{itemize}
    \item \textbf{Analizzabilità}: capacità di essere analizzato;
    \item \textbf{Modificabilità}: capacità di essere modificato;
    \item \textbf{Stabilità}: capacità di evitare effetti indesiderati;
    \item \textbf{Testabilità}: capacità di essere testato.
\end{itemize}

\subsubsection{Portabilità}
La portabilità misura la capacità del software di essere trasferito da un ambiente di lavoro a un altro.
\begin{itemize}
    \item \textbf{Adattabilità}: capacità di essere adattato;
    \item \textbf{Installabilità}: capacità di essere installato;
    \item \textbf{Sostituibilità}: capacità di essere sostituito.
\end{itemize}